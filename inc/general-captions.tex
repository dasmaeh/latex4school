% A nasty hack to ensure compatibility between float(s) and komascript!
% better make use of tocbasic and clean out this file (someday)...
\usepackage{scrhack}


% Description: extents the float mechanism of LaTeX and
%              provides macros for precise placement of 
%              figures, tables and captions.
%              works well together with the caption pack.
% load before: caption 
% Doc: floatrow.pdf 
\usepackage{floatrow, fr-fancy}

% Description: The caption package offers customization
%              of captions in floating environments such
%              figure and table and cooperates with many 
%              other packages.
% Doc: caption.pdf (Required v3.2 or newer)
\usepackage{caption}

%% subfig ist NOT recommended, use subcaption instead
%% Incompatible: 
%% - loads package capt-of. Loading of 'capt-of' afterwards will fail therefor
%% - subcaption
%% loads: caption
%% Doc: subfig.pdf
%\usepackage{subfig} 

% Description: subcaption supports typesetting of sub-captions
%             (by using the the sub-caption feature of the caption package).
% incompatible: subfig
% Doc: subcaption.pdf
\IfPackageNotLoaded{subfig}{
  % load after caption package
  \usepackage{subcaption}
}

% Description: provides a margincap environment for putting 
%              captions into the outer document margin with 
%              either a top or bottom alignment.
% Doc: mcaption.pdf
% \usepackage[
%   top, %  vertical caption alignment (top, bottom)
% ]{mcaption}

% Description: provides two new environments, sidewaystable and sidewaysfigure,
%              and further commands to rotate content.
% Doc: rotating.pdf
\usepackage[figuresright]{rotating}

\IfPackageLoaded{amsmath}{
% Numbering of figures and table in each chapter
% \numberwithin{figure}{chapter}
% \numberwithin{table}{chapter}
}

% Style of captions and subcaptions (and subfig)
\IfPackageLoaded{caption}{%
% Style of captions
\DeclareCaptionStyle{captionStyleTemplateDefault}
[ % single line captions
  justification = centering
]
{ % multiline captions
% -- Formatting
  format    = plain,  % plain, hang
  indention  = 0em,   % indention of text 
  labelformat = default,% default, empty, simple, brace, parens
  labelsep   = colon,  % none, colon, period, space, quad, newline, endash
  textformat  = simple, % simple, period
% -- Justification
  justification = justified, %RaggedRight, justified, centering
  singlelinecheck = true, % false (true=ignore justification setting in 
%single line)
% -- Fonts
  labelfont  = {small,bf},
  textfont   = {small,rm},
% valid values:
% scriptsize, footnotesize, small, normalsize, large, Large
% normalfont, ip, it, sl, sc, md, bf, rm, sf, tt
% singlespacing, onehalfspacing, doublespacing
% normalcolor, color=<...>
%
% -- Margins and further paragraph options
  margin = 10pt, %.1\textwidth,
  % width=.8\linewidth,
% -- Skips
  skip    = 10pt, % vertical space between the caption and the figure
  position = auto, % top, auto, bottom
% -- Lists
  % list=no, % suppress any entry to list of figure 
  listformat = subsimple, % empty, simple, parens, subsimple, subparens
% -- Names & Numbering
  % figurename = Abb. %
  % tablename  = Tab. %
  % listfigurename=
  % listtablename=
  % figurewithin=chapter
  % tablewithin=chapter
%-- hyperref related options
  hypcap=true, % (true, false) 
  % true=all hyperlink anchors are placed at the 
  % beginning of the (floating) environment
  %
  hypcapspace=0.5\baselineskip
}

% apply caption style
\captionsetup{
  style = captionStyleTemplateDefault % base
}

% Predefinded skip setup for different floats
\captionsetup[table]{position=bottom}
\floatsetup[table]{capposition=below}
\captionsetup[figure]{position=bottom}

\newcommand\FigureAbbrevition{Fig.}
\IfDefined{iflanguage}{%
  \iflanguage{ngerman}{%
    \renewcommand\FigureAbbrevition{Abb.}
  }{}
}

\DeclareCaptionStyle{captionStyleTemplateShortDefault}{%
  style=captionStyleTemplateDefault,
  name=\FigureAbbrevition,
  indention=0pt,
  justification=RaggedRight
}

% Short Names 
\IfDefined{wrapfigure}{%
  \captionsetup[wrapfigure]{style=captionStyleTemplateShortDefault}}
\IfDefined{wrapfloat}{%
  \captionsetup[wrapfloat]{style=captionStyleTemplateShortDefault}}
\IfDefined{floatingfigure}{%
  \captionsetup[floatingfigure]{style=captionStyleTemplateShortDefault}}
\IfDefined{margincap}{%
  \IfDefined{preto}{\preto\margincap{
  \captionsetup{style=captionStyleTemplateShortDefault}}}}
  % see http://tex.stackexchange.com/questions/37721/captionsetup-for-margin-caption
  % for an explanation of the extra code.
  %
} % end \IfPackageLoaded{caption}

% options for subcaptions
\IfPackageLoaded{subcaption}{
  \captionsetup[sub]{ %
    style = captionStyleTemplateDefault, % base
    labelfont  = {footnotesize,bf},
    textfont   = {footnotesize,rm},
    justification = RaggedRight, %RaggedRight, justified, centering
    skip=6pt,
    margin=5pt,
    labelformat = simple,% default, empty, simple, brace, parens
    labelsep    = space,
    list=false,
    hypcap=false
  }
  % make subcaptions be referenced as 5.3(b)
  \renewcommand\thesubfigure{(\alph{subfigure})} 
}

% style options for subfig
\IfPackageLoaded{caption}{%
 \IfPackageLoaded{subfig}{%
  \captionsetup[subfloat]{%
   style = captionStyleTemplateDefault, % base
   skip=6pt,
   margin=5pt,
   labelformat = parens,% default, empty, simple, brace
   labelsep    = space,
   list=false,
   hypcap=false
  }
 } % end \IfPackageLoaded{subfig}
} % end \IfPackageLoaded{caption}
\captionsetup[figure]{style=captionStyleTemplateShortDefault}
%\input{preamble/style-caption.tex}

% Style of figure placement with floatrow
\IfPackageLoaded{floatrow}{%

%\floatsetup[table]{style=plaintop}

\DeclareFloatStyle{TemplateFloatStyleBoxed}%
   {style=Boxed,frameset={\fboxrule1pt\fboxsep12pt}}
   
\DeclareFloatVCode{grayruleabove}%
   {{\color{gray}\par\rule\hsize{2.8pt}\vskip4pt\par}}
   
\DeclareFloatVCode{grayrulebelow}%
   {{\color{gray}\par\vskip4pt\rule\hsize{2.8pt}}}
   
\DeclareColorBox{TemplateFloatColorBoxStyle}%
   {\fcolorbox{gray}{white}}
   
\DeclareObjectSet{centering}{\centering}

\DeclareMarginSet{center}%
   {\setfloatmargins{\hfil}{\hfil}}
   
\DeclareMarginSet{hangleft}%
   {\setfloatmargins{\hskip-\marginparwidth\hskip-\marginparsep}{\hfil}}
   
\DeclareFloatSeparators{marginparsep}%
   {\hskip\marginparsep}

\floatsetup{%
   %% style
   style={%
      plain % Standard LaTeX
      % plaintop % puts captions above float object's contents
      % Plaintop % Capitalized form of plaintop
      % ruled
      % Ruled
      % boxed
      % Boxed
      % BOXED
      % shadowbox
      % Shadowbox
      % SHADOWBOX
      % Doublebox
      % DOUBLEBOX
      % wshadowbox
      % Wshadowbox
      % WSHADOWBOX
   },%
   %%% --- Font --
   % uses caption-package formats
   % font=
   % footfont=
   %%% --- Position of Caption ---
%   capposition=top, % caption above object
%   %% caption above object and also aligned by top line in float row.
%   capposition=TOP, 
%   capposition=bottom, % caption below object
%   capposition=beside, % caption beside object.
%   %
%   %%% --- Position of Beside Caption ---
%   %% caption is printed to the left side of object
%   capbesideposition=left, 
%   %% caption is printed to the right side of object;
%   capbesideposition=right, 
%   % caption is printed in binding side of page if
%   % twoside option switched on in document class and key
%   % facing=yes is used; in oneside option of document
%   % (or key facing=no is used), caption is printed at the left side;
%   capbesideposition=inside,
%   capbesideposition=outside,
%   % least popular option: caption printed in outer side of page
%   % if twoside option switched on in document class and key
%   % facing=yes is used; in oneside option of document
%   % (or key facing=no is used), caption is printed at the right side.
%   capbesideposition=top, % caption aligned to the top of object;
%   capbesideposition=bottom, % caption aligned to the bottom of object;
%   capbesideposition=center, % caption aligned to the center of object.
%   %
%   capbesidewidth=4cm, % Defines width of beside caption.
%   floatwidth=7cm, % Defines width of objects
%   capbesideframe=no, % Align Caption at frame, not text
   %
   footposition=default, % if caption above float object foot material is placed
                         % below float object, otherwise below caption;
%   footposition=caption, % always placed below caption;
%   footposition=bottom,  % always placed at the bottom of float box.
   %
   %%% --- Vertical Alignment of Float Elements ---
   %% - heightadjust ----
   heightadjust={%
      %all, % adjust both caption and object heights
                     % (e.g. for styles ruled, Ruled and BOXED);
      % caption, % adjust caption heights (e.g. for Plaintop style);
      object, % adjust object heights (e.g. for Boxed style);
      % none, % nothing to be adjusted (the plain style);
      % nocaption, % no adjusting for captions;
      % noobject, % no adjusting for objects;
   },%
   %
   %% - valign ---
   % valign=t, % aligns objects by top line;
   % valign=c, % aligns objects by center line
   valign=b, % aligns objects by bottom line;
   % valign=s, % stretches objects by full height (if it is possible).
   %%% --- Facing Layout ---
   facing=yes, % different layout for even and odd pages in if twoside is on
   %%% --- Object Settings ---
   %% - objectset: Defines justification of float object (float contents).
   % objectset=justified,    %
   objectset=centering,    %
   % objectset=raggedright,  %
   % objectset=RaggedRight,  %
   %%% --- Defining Float Margins ---
   %% - margins: ????
   margins=centering,   %
   % margins=raggedright, %
   % margins=raggedleft,  %
   %%% --- Defining Float Separators ---
   % horizontal skip = \columnsep (default for both keys);
    floatrowsep=columnsep, 
   % floatrowsep=quad,  % horizontal skip = 1 em;
   % floatrowsep=qquad, % horizontal skip = 2 em;
   % floatrowsep=hfil,  % like \hfil
   % floatrowsep=hfill, % like \hfill
   % floatrowsep=none,  % empty separator
   %
   % horizontal skip = \columnsep (default for both keys);
   capbesidesep=columnsep, 
   % capbesidesep=quad,  % horizontal skip = 1 em;
   % capbesidesep=qquad, % horizontal skip = 2 em;
   % capbesidesep=hfil,  % like \hfil
   % capbesidesep=hfill, % like \hfill
   % capbesidesep=none,  % empty separator
   %%% --- Defining Float Rules/Skips ---
   %% - precode:     above float box
   precode={
      none %
      % thickrule %
      % rule %
      % lowrule %
      % captionskip
   },%
   %% - rowprecode:  above alone float box
   rowprecode={
      none %
      % thickrule %
      % rule %
      % lowrule %
      % captionskip
   },%
   %% - midcode:     between caption above/below and float object.
   midcode={%
      %none %
      % thickrule %
      % rule %
      % lowrule %
      captionskip
   },%
   %% - postcode:    below float box
   postcode={%
      none %
      % thickrule %
      % rule %
      % lowrule %
      % captionskip
   },%
   %% - rowpostcode: below alone float box
   rowpostcode={%
      none %
      % thickrule %
      % rule %
      % lowrule %
      % captionskip
   },%
   %%% --- Defining Float Frames ---
%   framestyle={%
%      % fbox %
%      colorbox %
%      % doublebox %
%      % shadowbox %
%      % wshadowbox %
%   },
   %% - frameset: The parameters for chosen frame
   % frameset={\fboxrule1pt\fboxsep12pt},
%   framearound={%
%      object % float object contents
%      % all % full float box
%   },
   framefit=yes, % fit frame to whatever is set
   %%% --- Settings for Colored Frames ---
   % Predefinded ColorBox (\DeclareColorBox)
%   colorframeset=TemplateFloatColorBoxStyle,
   %%% --- Defining Float Skips ---
   captionskip=5pt,
   footskip=\skip\footins,
   %%% --- Defining Float Footnote Rule's Style ---
   % Defines type of footnote rule for footnotes inside floating environment.
   footnoterule={
      normal   % standard LaTeX definition
      % limited  % standard LaTeX definition, max width of footnote \frulemax
      % fullsize % rule to full current text width.
      % none     % Absent rule.
   },
   %%% --- Managing Floats with [H] Placement Option ---
   % doublefloataswide=true, % ???
   % floatHaslist=false, % only true for backward compatibility
}


\floatsetup[FloatStyleCaptionMargin]{
  margins=hangleft,
  floatwidth=\textwidth,
  capposition=beside,
  capbesideposition=left,
  capbesideframe=no,
  capbesidewidth=\marginparwidth,
  capbesidesep=marginparsep,
  framestyle=framefit=yes,
}

%%% Replacement of <float> Package
%\DeclareNewFloatType{%
%   placement={%
%      tbh % any of t,b,h,p
%   },%
%   name={
%      % Defines the name of environment in the caption label.
%   },%
%   fileext={
%       % Defines extension of the file in which gathered list of floats.
%   }
%   within={% Reset caption within...
%      % nothing = do not reset ever
%      section % also section/chapter/part
%   },%
%   relatedcapstyle=yes % yes/no, related to \captionsetup
%}%

}% end if 
