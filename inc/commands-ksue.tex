% eigene Commands 

% Rahmenumgebungen (für Skripte)
\theoremstyle{example}
\newtheoremstyle{definition}% name of the style to be used
  {1.ex}% measure of space to leave above the theorem. E.g.: 3pt
  {1.5ex}% measure of space to leave below the theorem. E.g.: 3pt
  {}% name of font to use in the body of the theorem
  {}% measure of space to indent
  {\bfseries}% name of head font
  {:}% punctuation between head and body
  { }% space after theorem head; " " = normal interword space
  {}% Manually specify head

\theoremstyle{definition} 
\newtheorem*{beispiel*}{Beispiel}

% --| Merkkasten | ---------------------------------------------------
\mdfdefinestyle{merkestyle}{%
  linecolor=red, innerlinewidth=2pt, outerlinewidth=1pt,%
  middlelinecolor=white, middlelinewidth=4pt,%
  hidealllines=true, leftline=true,%
  backgroundcolor=verylightgray,%
  innertopmargin=\topskip, innerbottommargin=\topskip,%
  %skipbelow=0%
}

%\newmdenv[style=merkestyle]{merke}
%\mdtheorem[style=merkestyle]{merketitled}{}

% --| Experiment | ---------------------------------------------------

% Liste der Experimente:
\newcommand{\listexpname}{Liste der Versuche}
\newlistof[chapter]{experiment}{exp}{\listexpname}
% Liste der GFS-Themen
\newcommand{\listgfsname}{Liste der GFS-Themen}
\newlistof[chapter]{gfscount}{gfs}{\listgfsname}

% Styles für Experimentumgebungen, sollten vielleicht besser nach -> styles/
\mdfdefinestyle{expstyle_lv}{%
  frametitle=true,frametitlebackgroundcolor=verylightgray, frametitlerule=true, frametitlerulecolor=darkgray,%
  frametitleaboveskip=2pt,frametitlebelowskip=0pt,frametitlefont=\sffamily,%
  linecolor=orange, innerlinewidth=2pt, outerlinewidth=1pt,%
  middlelinecolor=white, middlelinewidth=4pt,%
  hidealllines=true, leftline=true, %  
}

\mdfdefinestyle{expstyle_sv}{%
  frametitleaboveskip=2pt,frametitlebelowskip=0pt,frametitlefont=\sffamily,%
  frametitle=true,frametitlebackgroundcolor=verylightgray, frametitlerule=true, frametitlerulecolor=darkgray,%
  linecolor=cyan, innerlinewidth=2pt, outerlinewidth=1pt,%
  middlelinecolor=white, middlelinewidth=4pt,%
  hidealllines=true, leftline=true, %
}

% Für theoretische Versuche
\mdfdefinestyle{expstyle_tv}{%
  frametitleaboveskip=2pt,frametitlebelowskip=0pt,frametitlefont=\sffamily,%
  frametitle=true,frametitlebackgroundcolor=verylightgray, frametitlerule=true, frametitlerulecolor=darkgray,%
  linecolor=red, innerlinewidth=2pt, outerlinewidth=1pt,%
  middlelinecolor=white, middlelinewidth=4pt,%
  hidealllines=true, leftline=true, %
}

\newtheoremstyle{exp}%
  {}{}%
  {}{}%
  {\bfseries}{}%  % Note that final punctuation is omitted.
  {\newline}{}
  
\newtheoremstyle{script}
  {}{}%
  {}{}%
  {\bfseries}{}%  % Note that final punctuation is omitted.
  {\newline}{\thmname{#1}\thmnumber{ #2}:\thmnote{ #3}}

% Zähler für Versuche: 
\newcounter{exp}[section]
% Umgebung für Lehrerversuche:
\newenvironment{experiment}[2][]{%
  \stepcounter{exp}%
   \def\expcolor{gray!40}
   \def\theoretical{}
   \def\lv{lv}\def\sv{sv}\def\tv{tv}
   \def\test{#1}
   % Falls LV
   \ifx\test\lv
      \def\expcolor{orange!40}
   \fi
   % Falls SV
   \ifx\test\sv
      \def\expcolor{blue!40}
   \fi
   % Falls TV
   \ifx\test\tv
      \def\expcolor{red!40}
      \def\theoretical{~(theoretisch)}
   \fi
     \ifstrempty{#2}%
      {
          \mdfsetup{%
        frametitle={%
          \tikz[baseline=(current bounding box.east), outer sep=0pt]
          \node[anchor=east,rectangle,fill=\expcolor]
          {\strut \sffamily V\textsubscript{\thelv}};}
        }%
      }%
      {
        \mdfsetup{%
        frametitle={%
          \tikz[baseline=(current bounding box.east),outer sep=0pt]
          \node[anchor=east,rectangle,fill=\expcolor]
          {
          \strut \normalfont\sffamily V\textsubscript{\theexp}:~#2\theoretical
          };
          }
        }%
      }%
    \mdfsetup{innertopmargin=0pt,linecolor=\expcolor,%
      linewidth=2pt,topline=true, frametitleaboveskip=\dimexpr-\ht\strutbox\relax,
      }
    \begin{mdframed}[]\relax%
    }{\end{mdframed}}

% Framed theorem environment für Lehrerversuche
% \mdtheorem[style=expstyle_lv]{lv}{V\refstepcounter{experiment}\addcontentsline{exp}{experiment}{\textbf{LV \protect{\numberline{\theexperiment}}:} #1}}
% Framed theorem environment für Schülerversuche
% \mdtheorem[style=expstyle_sv]{sv}{V\refstepcounter{experiment}\addcontentsline{exp}{experiment}{\textbf{SV \protect{\numberline{\theexperiment}}:} #1}}
% Framed theorem environment für Theorieversuche
% \mdtheorem[style=expstyle_tv]{tv}{V\refstepcounter{experiment} (theoretisch) \addcontentsline{exp}{experiment}{\textbf{TV \protect{\numberline{\theexperiment}}:} #1}}


\mdtheorem[style=gfsstyle]{gfs}{GFS-Thema\refstepcounter{gfscount}\addcontentsline{gfs}{gfscount}{\textbf{GFS \protect{\numberline{\thegfscount}}:} #1}}

% Theorem environments für Durchführung, Beobachtung, Erklärung
\theoremstyle{exp}
\newtheorem*{expdf}{Df:}
\newtheorem*{expbe}{B:}
\newtheorem*{exper}{E:}
% Umgebung für Impulse und Arbeitsaufträge
\newenvironment{auftrag}[1][]{%
     \ifstrempty{#1}%
      {
          \mdfsetup{%
        frametitle={%
          \tikz[baseline=(current bounding box.east), outer sep=0pt]
          \node[anchor=east,rectangle,fill=gray!40]
          {\strut \sffamily};}
        }%
      }%
      {
        \mdfsetup{%
        frametitle={%
          \tikz[baseline=(current bounding box.east),outer sep=0pt]
          \node[anchor=east,rectangle,fill=gray!40]
          {
          \strut \normalfont\sffamily #1:
          };
          }
        }%
      }%
    \mdfsetup{innertopmargin=0pt,linecolor=gray!40,%
      linewidth=2pt,topline=true, frametitleaboveskip=\dimexpr-\ht\strutbox\relax,
      }
    \begin{mdframed}[]\relax%
    }{\end{mdframed}}
    
% Umgebung für Merkkästen
\newenvironment{merke}[1][]{%
     \ifstrempty{#1}%
      {
          \mdfsetup{%
        frametitle={%
          \tikz[baseline=(current bounding box.east), outer sep=0pt]
          \node[anchor=east,rectangle,fill=red!40]
          {\strut \sffamily};}
        }%
      }%
      {
        \mdfsetup{%
        frametitle={%
          \tikz[baseline=(current bounding box.east),outer sep=0pt]
          \node[anchor=east,rectangle,fill=red!40]
          {
          \strut \normalfont\sffamily #1:
          };
          }
        }%
      }%
    \mdfsetup{innertopmargin=0pt,linecolor=red!40,%
      linewidth=2pt,topline=true, frametitleaboveskip=\dimexpr-\ht\strutbox\relax,
      }
    \begin{mdframed}[]\relax%
    }{\end{mdframed}}
