% Tabellen
\usepackage{booktabs}
\usepackage{tabu}
\usepackage{tabularx}
\usepackage{tabulary}
\usepackage{longtable}
\usepackage{multirow}
%, colortbl, array}

% Zeilenhöhe (Für AB), allegmein evtl. weniger
\def\arraystretch{1.25}%
% for Package tabu
% \IfDefined{tabulinesep}{%
%   \tabulinesep=5pt
% }
% % Abstand Inhalt-Linie muss 0 sein für rowcolor
\setlength{\belowrulesep}{0pt}
\setlength{\aboverulesep}{0pt}

\def\arraystretch{1.5}%  1 is the default, change whatever you need

%\newcolumntype{C}[1]{>{\Centering\arraybackslash\hspace{0pt}}p{#1}}

% Define new column types only if they are not yet defined
\IfDefined{RaggedLeft}{
  %% centered (Z):
  \IfColumntypeDefined{Z}{}
    {\newcolumntype{Z}{>{\Centering\arraybackslash\hspace{0pt}}X}}
  %% right (X):
  \IfColumntypeDefined{Y}{}
    {\newcolumntype{Y}{>{\RaggedLeft\arraybackslash\hspace{0pt}}X}}
  %% left (X):
  \IfColumntypeDefined{W}{}
    {\newcolumntype{W}{>{\RaggedRight\arraybackslash\hspace{0pt}}X}}
  %% left (p):
  \IfColumntypeDefined{L}{}
    {\newcolumntype{L}[1]{>{\RaggedRight\arraybackslash\hspace{0pt}}p{#1}}}
  %% right (p):
  \IfColumntypeDefined{R}{}
    {\newcolumntype{R}[1]{>{\RaggedLeft\arraybackslash\hspace{0pt}}p{#1}}}
  %% centered (p):
  \IfColumntypeDefined{C}{}
    {\newcolumntype{C}[1]{>{\Centering\arraybackslash\hspace{0pt}}p{#1}}}
%    \newcolumntype{C}{>{\centering\arraybackslash}X}
}
%
